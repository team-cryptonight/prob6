\documentclass{article}
\usepackage{amsmath}
\usepackage{amssymb}
\usepackage{amsthm}
\usepackage{kotex}
\usepackage[backend=biber]{biblatex}
\usepackage{hyperref}

\addbibresource{citation.bib}
\setmainhangulfont[ItalicFont={*},ItalicFeatures={FakeSlant=.22}]{Noto Serif CJK KR}

\newtheorem{definition}{정의}
\newtheorem{theorem}{정리}

\newcommand{\adv}{\mathbf{Adv}}

\title{6번 문제 풀이}
\author{Crypto Night}

\begin{document}
  \maketitle

  \section{Indifferentiability Proof}
  \(b, c, r, n \in \mathbb{N}, b = c + r\)이고, \(\mathcal{P}\)를 \(b\)-bit permutation, \(\mathrm{pad}\)를 \(r\)비트 블록으로 변환하는 injective padding이라고 하자. Padding된 마지막 블록은 non-zero여야 한다.
  (TODO: generalized sponge construction \(\mathcal{S}[\mathcal{P}, \mathrm{pad}, r]\)  정의 완성)

  한편, 문제에서 설명하는 해시 함수를 \(\mathcal{S}'[\mathcal{P}, \mathrm{pad}, r_1, r_2, r_3]\)라고 정의하면, 다음이 성립한다.

  \begin{theorem}[두 construction의 관계]
    \(10^*\)-padding \(\mathrm{pad10^*}\)과 \(10^*1\)-padding \(\mathrm{pad10^*1}\)을 정의할 때, 모든 bitrate \(0 \leq r_1, r_2, r_3 \leq r_\mathrm{max}\)에 대해 함수 \(I[r_1, r_2, r_\mathrm{max}], O[r_3, r_\mathrm{max}]\)가 존재하고, 다음 관계를 만족한다.

    \begin{align*}
      \mathcal{S}'[\mathcal{P}, \mathrm{pad10^*1}, r_1, r_2, r_3] = O[r_3, r_\mathrm{max}] \circ \mathcal{S}[\mathcal{P}, \mathrm{pad10^*}, r_\mathrm{max}, r_3] \circ I[r_1, r_2, r_\mathrm{max}]
    \end{align*}

    여기에서 \(f \circ g\)는 함수의 합성을 의미한다.
  \end{theorem}
  \begin{proof}
    (TODO: 증명 완성)
  \end{proof}

  \begin{theorem}[Indifferentiability]
    \(\mathcal{S}'[h, \mathrm{pad10^*1}, r_1, r_2, r_3]\)를 random oracle과 구분하는 것과 \(\mathcal{S}[h, \mathrm{pad10^*}, r_\mathrm{max}, r_3]\)를 random oracle과 구분하는 advantage는 같다.
  \end{theorem}
  \begin{proof}
    (TODO: 증명 완성)
  \end{proof}

  \section{Security Model}
  Adversary \(\mathcal{A}\)를 확률적 알고리즘이라고 가정하고, 우리의 해시 함수 \(\mathcal{H}: \{0, 1\}^* \to \{0, 1\}^n\)는 permutation \(\mathcal{P}\)에 기반한다고 가정하자. \(\mathcal{A}\)는 \(q\)번까지 \(\mathcal{P}, \mathcal{P}^{-1}\)에 접근하는 query가 가능하며, 같은 query를 두 번 이상 하지 않는다고 가정하자.

  \section{Preimage Resistance Lower Bound}
  우리는 여기에서 everywhere preimage resistance에 집중할 것이다.\cite{rogaway_cryptographic_2004} Random oracle \(\mathcal{R}\)에서 다음이 성립한다.\cite{andreeva_security_2011}

  \begin{align}\label{epre_adv_bound}
    \adv^\text{epre}_\mathcal{H}(q) \leq \adv^\text{pro}_\mathcal{H}(q) + \adv^\text{epre}_\mathcal{R}(q)
  \end{align}

  여기서 \(\adv^\text{pro}_\mathcal{R}(q)\)는 primitive \(\pi\)를 기반으로 하는 해시함수 \((\mathcal{H}, \pi)\)를 어떤 simulator \(S\)에 대해 random oracle \((\mathcal{R}, S)\)로부터 구분하는 advantage로 정의된다. \(\adv^\text{epre}_\mathcal{R}=q/2^n\)이므로,

  (나머지는 더 공부한다음에 완성하겠읍니다...)

  \section{Collision Resistance Lower Bound}
  (\ref{epre_adv_bound})의 식과 비슷하게, 다음이 성립한다.

  \begin{align*}
    \adv^\text{coll}_\mathcal{H}(q) \leq \adv^\text{pro}_\mathcal{H}(q) + \adv^\text{coll}_\mathcal{R}(q)
  \end{align*}

  (나머지는 더 공부한다음에 완성하겠읍니다...)

  \section{Second Preimage Resistance Lower Bound}
  Collision resistance는 second preimage reistance를 함의함이 증명되어 있다.\cite{rogaway_cryptographic_2004} 다르게 말하면, 다음이 성립한다.

  \begin{align*}
    \adv^\text{sec}_\mathcal{H}(q) \leq \adv^\text{coll}_\mathcal{H}(q)
  \end{align*}

  \printbibliography
\end{document}
